\chapter{Discussion}
This study investigated the utilization, a potential use case of LoRAs among Flemish architects.

% \section{How the architects used the tool}
% In the user study, there were several occasions where the architect requested a modification on the provided image, something this tool did not account for since it was primarily built to generate new images and explore ideas.\\

% \section{Aspects of the generated images}
% FLUX.1 [dev] in combination with LoRA's, trained on a Flemish architect's own mannerisms, outputs images that are more architecturally interesting to the architect, compared to images generated by FLUX alone. The images with LoRA showed a design intent behind them (storytelling, special roof shape reflecting to the spaces below), and showed interesting compositional qualities. This is shown by the impression of both architects: they thought the images created with LoRA seemed more 'thought out'. The images without lora often portray 'modern buildings', in a very general style, of which 'there are so many already'.\\
% So, even when not considering the concepts that the lora's were explicitly trained on, 
% The images without LoRA can sometimes give better results, such as in the first session where the structure of the buildings felt lighter. These cases are scarce however, compared to cases where the LoRA gives more interesting images.\\
% The architects, overall, thought that the images created by the LoRA's were more interesting compared to those generated without LoRA. This is illustrated by the fact that the architects

% \section{Use cases}
% The images created by help of the LoRA's make use cases possible where architects might use these models to gain inspiration in their design process. In this section, two possible ways are explained in which the architects could use these images to their advantage.\\
% \subsection{ideation}
% When LAVA architecten hires new people, the biggest problem they often face is the 'white page'. These images could provide the architects with a few starting points. In this way, the images pose a challenge to the architect: should he incorporate these ideas or not? In this process, the architect naturally comes closer to what he wants.\\
% The images can help architects to come out of the fixation problem in architectural design, and can challenge them to question things they thought of as trivial.
% \subsection{verification of ideas}

\section{}


Typically, the AI model does not produce the required outcomes accurately on the first try.