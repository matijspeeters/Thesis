\chapter{Conclusion}

This thesis delved into the topic of fine-tuning diffusion models, to generate images which are tied closer to an office's own style.\\
Placing itself in the context of Flemish architectural practice, the idea to train on an office's style did not turn out very well, since Flemish offices often don't portray a 'house style'. Rather, most Belgian, and especially Flemish practices love the idea of creating more context-specific architecture, heavily based on the spatial and social context of the site.\\
\\
The core idea of this thesis, therefore, was to define a new concept, 'mannerisms', that encompass simple concepts that architects might use in their designs, to make the process of training a LoRA easier and quicker. In this way, this study's purpose was to find out whether these LoRA models could be useful, specifically to Flemish architects.\\
\\
This study found that some concepts work better than others to train LoRAs on; namely, more abstract concepts work better, and materialistic concepts work the best among the 6 LoRAs trained in this thesis. The other 4 concepts, being more compositional or formal, were easier to prompt using the base model FLUX.1 and thus don't work very well in a LoRA model. For such purposes, the use of ControlNet or similar geometry-conserving methods could be a better solution.\\
\\
It also uncovered some caviats that AI-tools like these still have to deal with; there are some requests that the architects pose, that the tool was not built for. 